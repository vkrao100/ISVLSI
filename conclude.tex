\vspace{-0.1in} 
\section{Conclusion}
\label{sec:conclude}

We model the rectification problem using concepts from
algebraic geometry and use symbolic computer algebra algorithms to compute the rectification function $U(X_{PI})$. Using the model presented in \cite{Armin2017ColumnWiseVO}, we obtain a system of polynomials $F = \{f_1,\dots,f_s\}$ to describe the gates of the circuit over the field of rational numbers $\Q$. A net $x_i$ in a buggy circuit is found to admit single-fix rectification, confirming the existence of a polynomial $x_i = U(X_{PI})$ in primary input variables that can rectify the circuit at net $x_i$. The problem of computing $U(X_{PI})$ is modeled as an extended ideal membership testing problem, which is
solved using the \Grobner Basis algorithm. This paper investigates the nature of the rectification function, by stating the conditions under which the rectification polynomial we compute is a Boolean polynomial, i.e., it maps from $\{0,1\}^{|X_{PI}|}\rightarrow \{0,1\}$. We also introduce the concept of don't cares of the rectification function in the computer algebra setup. This paper describes the procedure to synthesize a logic circuit from the rectification polynomial we compute. By conducting experiments on integer array multipliers, we demonstrate the efficiency of our approach. 
