\section{Introduction}

Past few years have witnessed significant advances in automatic
formal verification of arithmetic circuits. Given a polynomial
specification (\spec $f$) and an implementation circuit $C$, 
techniques have been researched that formally prove that $C$
implements $f$, or ascertain the 
presence of a bug in the design. These techniques use an {\it
  algebraic (i.e. polynomial algebra) model} as a core computational
engine for verification, and have demonstrated
successful verification for integer arithmetic circuits
\cite{kauffman2017} \cite{Armin2017ColumnWiseVO} \cite{PolyCleaner}
\cite{maciej:2016:1} \cite{Cunxi:TCAD_2018_9}, integer modulo
arithmetic circuits \cite{wienand:cav08}, and also for finite field
circuits \cite{lv:tcad2013} \cite{pruss:tcad} -- all with large
operand sizes.


When formal verification of arithmetic circuits detects the presence
of a bug in the design, then it is required to perform
post-verification debugging and {\bf rectification} of the buggy
implementation so that the rectified circuit matches the given
specification. In such a setting, it is desirable to compute
rectification functions at internal nets, without
resynthesizing/redesigning the entire circuit -- i.e. the problem of {\it
  partial synthesis}. The partial synthesis problem is more involved
than that of verification, as the former is a quantification problem
whereas the latter is a decision problem. Rectification 
requires to first identify a set of nets where the circuit can be
rectified, and then to subsequently compute corresponding
rectification functions at those nets.  While this problem of has been
addressed
%(for debugging as well as for engineering change orders)
for random-logic circuits \cite{Sadowska:DAC95} \cite{Huang:DAC2011}
\cite{fujita:2015} \cite{scholl:2}, as well as recently for finite
field circuits \cite{Utkarsh:vlsisoc18} \cite{utkarsh:fmcad18}, the
problem has not been satisfactorily addressed for rectification of
integer arithmetic circuits. This paper addresses the
problem of {\it computing and synthesizing rectification functions for
  buggy integer arithmetic circuits}. 

{\it Problem Statement:} We are given an integer arithmetic circuit
$C$ and a polynomial specification $f$. Formal verification is
performed to verify $C$ against $f$ (say, using the techniques of
\cite{Armin2017ColumnWiseVO}). We assume that verification determines
that $C$ is buggy, i.e. $C$ {\underline{does not}} implement $f$. No
assumptions are made about the type, nature or the number of bugs in
$C$. Assume further that post-verification debugging has been
performed and {\it it has been determined that $C$ can be rectified at
  a single-net $x_i$}. In other words, we assume that $C$ admits {\it
  single-fix rectification} at net $x_i$. Our {\it objective} is
to compute and synthesize a Boolean function $U(X_{PI})$ in
terms of primary input variables of the circuit ($X_{PI}$) such that
modifying the circuit at $x_i = U(X_{PI})$ rectifies $C$ to match
$f$. 

\textit{Related Previous Work:} 
Automated diagnosis and rectification of digital circuits
was addressed in \cite{Madre:ICCAD89}, and the technique was extended
in \cite{Sadowska:DAC95} for synthesizing Engineering Change Order
(ECO) patches - an analogous problem to rectification. Once
rectification is deemed feasible, the problem of finding the 
rectification function has been considered as partial synthesis of an
unknown component $U$. The approach \cite{fujita:2015} formulates the
computation of $U$ as a quantified Boolean formula (QBF), and solves
it using incremental and iterative SAT solving. Similarly, the authors
in~\cite{scholl:2} present a QBF formulation for answering whether a
partial implementation (with block-boxes $U$) can be extended to a
complete design that models a given specification. As Craig
interpolation is an alternative to QBF solving, it has been presented
in \cite{Huang:DAC2011} for multi-fix rectification for ECO synthesis.  

%% The rectification problem has been addressed for finite field
%% arithmetic circuits in \cite{utkarsh:vlsisoc18} \cite{utkarsh:fmcad18}
%% and \cite{utkarsh:ets19}. 
The concept of Craig interpolation was extended to polynomials in
finite fields \cite{Utkarsh:vlsisoc18} \cite{utkarsh:ets19} and
application to rectification of finite field arithmetic circuits was
demonstrated. As Craig interpolation in finite fields is a
computationally hard problem, another rectification approach was
presented for finite field circuits in \cite{utkarsh:fmcad18} which
uses the (extended) \Grobner basis algorithm as a quantification
procedure to compute the rectification function. Rectification has
been attempted for integer arithmetic circuits \cite{farimah:2017:1}
\cite{Rolf:ISVLSI2018}. Unfortunately, the technique of
\cite{farimah:2017:1} was demonstrated to be incomplete in
\cite{utkarsh:fmcad18}, and \cite{Rolf:ISVLSI2018} is limited by the
fault model which only addresses the gate misplacement fault. 


\textit{Approach and Contributions:} We model the rectification problem
using concepts from algebraic geometry and use symbolic computer
algebra algorithms to compute the rectification function
$U(X_{PI})$. Using the model presented in
\cite{Armin2017ColumnWiseVO}, we describe the gates of $C$ with
a set of polynomials $F = \{f_1,\dots, f_s\}$ with integer
coefficients. Subsequently, the problem of computing $U(X_{PI})$ is
modeled as an extended ideal membership testing problem, which is
solved using the \Grobner basis algorithm \cite{gb_book}. 

Our work draws inspirations from \cite{kauffman2017}
\cite{Armin2017ColumnWiseVO} and \cite{utkarsh:fmcad18}. In
\cite{kauffman2017}\cite{Armin2017ColumnWiseVO}, the authors showed
that the verification of integer multipliers could be formulated and
solved over polynomial rings with coefficients from the field of
fractions $\Q$, as opposed to solving over integers $\Z$. As most
algebraic geometry results are valid over fields (and $\Z \subset
\Q$ is not a field), their approach leveraged theory as well as
efficient computational techniques over fields to solve the
verification problem. Also, \cite{utkarsh:fmcad18} shows how to
compute single-fix rectification functions $x_i = U(X_{PI})$ over
finite field circuits. {\it We use the polynomial model of
\cite{kauffman2017}   \cite{Armin2017ColumnWiseVO} to perform \Grobner
basis computations over $\Q$, and use the ideal membership approach
presented in Section VI in \cite{utkarsh:fmcad18} to compute the
rectification function.} 


Our approach assumes that it has been determined that the net $x_i$ in
$C$ admits single-fix rectification. This implies the {\it existence of a
Boolean function} $U: \{0,1\}^{|X_{PI}|}\rightarrow \{0,1\}$, such
that $x_i = U(X_{PI})$ rectifies the circuit. However, since we use a
polynomial model over $\Q$, application of the techniques of
\cite{utkarsh:fmcad18} computes $U$ as a polynomial function over
$\Q$, i.e. it may evaluate to non-Boolean values in $\Q$. Therefore,
one of the main contribution of our approach is that it investigates
the nature of the rectification polynomial, its relationship to don't
care conditions, and shows how to synthesize a rectification
polynomial $U$ for integer arithmetic circuits such that it maps as
$\{0,1\}^{|X_{PI}|} \rightarrow \{0,1\}$. Experiments demonstrate the
application of our approach to integer (array) multiplier circuits. 

Finally, the problem of ascertaining single-fix rectifiability at net
$x_i$ is not described in this paper (mostly due to lack of space). It
can be solved in a manner similar to Theorem V.1 in
\cite{utkarsh:fmcad18}. Similarly, multi-fix rectification is also
beyond the scope of this paper. 

{\it Paper Organization:} The paper is organized as follows: The
following section covers preliminary concepts. Sec. \ref{verif} covers
the verification setup, and Sec. \ref{rect} describes the rectification
theory and approach. Experiments are described in Sec. \ref{sec:exp},
and Sec. \ref{sec:conclude} concludes the paper. 
